% Options for packages loaded elsewhere
% Options for packages loaded elsewhere
\PassOptionsToPackage{unicode}{hyperref}
\PassOptionsToPackage{hyphens}{url}
\PassOptionsToPackage{dvipsnames,svgnames,x11names}{xcolor}
%
\documentclass[
  letterpaper,
  DIV=11,
  numbers=noendperiod]{scrartcl}
\usepackage{xcolor}
\usepackage{amsmath,amssymb}
\setcounter{secnumdepth}{-\maxdimen} % remove section numbering
\usepackage{iftex}
\ifPDFTeX
  \usepackage[T1]{fontenc}
  \usepackage[utf8]{inputenc}
  \usepackage{textcomp} % provide euro and other symbols
\else % if luatex or xetex
  \usepackage{unicode-math} % this also loads fontspec
  \defaultfontfeatures{Scale=MatchLowercase}
  \defaultfontfeatures[\rmfamily]{Ligatures=TeX,Scale=1}
\fi
\usepackage{lmodern}
\ifPDFTeX\else
  % xetex/luatex font selection
\fi
% Use upquote if available, for straight quotes in verbatim environments
\IfFileExists{upquote.sty}{\usepackage{upquote}}{}
\IfFileExists{microtype.sty}{% use microtype if available
  \usepackage[]{microtype}
  \UseMicrotypeSet[protrusion]{basicmath} % disable protrusion for tt fonts
}{}
\makeatletter
\@ifundefined{KOMAClassName}{% if non-KOMA class
  \IfFileExists{parskip.sty}{%
    \usepackage{parskip}
  }{% else
    \setlength{\parindent}{0pt}
    \setlength{\parskip}{6pt plus 2pt minus 1pt}}
}{% if KOMA class
  \KOMAoptions{parskip=half}}
\makeatother
% Make \paragraph and \subparagraph free-standing
\makeatletter
\ifx\paragraph\undefined\else
  \let\oldparagraph\paragraph
  \renewcommand{\paragraph}{
    \@ifstar
      \xxxParagraphStar
      \xxxParagraphNoStar
  }
  \newcommand{\xxxParagraphStar}[1]{\oldparagraph*{#1}\mbox{}}
  \newcommand{\xxxParagraphNoStar}[1]{\oldparagraph{#1}\mbox{}}
\fi
\ifx\subparagraph\undefined\else
  \let\oldsubparagraph\subparagraph
  \renewcommand{\subparagraph}{
    \@ifstar
      \xxxSubParagraphStar
      \xxxSubParagraphNoStar
  }
  \newcommand{\xxxSubParagraphStar}[1]{\oldsubparagraph*{#1}\mbox{}}
  \newcommand{\xxxSubParagraphNoStar}[1]{\oldsubparagraph{#1}\mbox{}}
\fi
\makeatother


\usepackage{longtable,booktabs,array}
\usepackage{calc} % for calculating minipage widths
% Correct order of tables after \paragraph or \subparagraph
\usepackage{etoolbox}
\makeatletter
\patchcmd\longtable{\par}{\if@noskipsec\mbox{}\fi\par}{}{}
\makeatother
% Allow footnotes in longtable head/foot
\IfFileExists{footnotehyper.sty}{\usepackage{footnotehyper}}{\usepackage{footnote}}
\makesavenoteenv{longtable}
\usepackage{graphicx}
\makeatletter
\newsavebox\pandoc@box
\newcommand*\pandocbounded[1]{% scales image to fit in text height/width
  \sbox\pandoc@box{#1}%
  \Gscale@div\@tempa{\textheight}{\dimexpr\ht\pandoc@box+\dp\pandoc@box\relax}%
  \Gscale@div\@tempb{\linewidth}{\wd\pandoc@box}%
  \ifdim\@tempb\p@<\@tempa\p@\let\@tempa\@tempb\fi% select the smaller of both
  \ifdim\@tempa\p@<\p@\scalebox{\@tempa}{\usebox\pandoc@box}%
  \else\usebox{\pandoc@box}%
  \fi%
}
% Set default figure placement to htbp
\def\fps@figure{htbp}
\makeatother





\setlength{\emergencystretch}{3em} % prevent overfull lines

\providecommand{\tightlist}{%
  \setlength{\itemsep}{0pt}\setlength{\parskip}{0pt}}



 


\usepackage{fvextra}
\DefineVerbatimEnvironment{Highlighting}{Verbatim}{
  commandchars=\\\{\},
  breaklines, breaknonspaceingroup, breakanywhere
}
\KOMAoption{captions}{tableheading}
\makeatletter
\@ifpackageloaded{tcolorbox}{}{\usepackage[skins,breakable]{tcolorbox}}
\@ifpackageloaded{fontawesome5}{}{\usepackage{fontawesome5}}
\definecolor{quarto-callout-color}{HTML}{909090}
\definecolor{quarto-callout-note-color}{HTML}{0758E5}
\definecolor{quarto-callout-important-color}{HTML}{CC1914}
\definecolor{quarto-callout-warning-color}{HTML}{EB9113}
\definecolor{quarto-callout-tip-color}{HTML}{00A047}
\definecolor{quarto-callout-caution-color}{HTML}{FC5300}
\definecolor{quarto-callout-color-frame}{HTML}{acacac}
\definecolor{quarto-callout-note-color-frame}{HTML}{4582ec}
\definecolor{quarto-callout-important-color-frame}{HTML}{d9534f}
\definecolor{quarto-callout-warning-color-frame}{HTML}{f0ad4e}
\definecolor{quarto-callout-tip-color-frame}{HTML}{02b875}
\definecolor{quarto-callout-caution-color-frame}{HTML}{fd7e14}
\makeatother
\makeatletter
\@ifpackageloaded{caption}{}{\usepackage{caption}}
\AtBeginDocument{%
\ifdefined\contentsname
  \renewcommand*\contentsname{Table of contents}
\else
  \newcommand\contentsname{Table of contents}
\fi
\ifdefined\listfigurename
  \renewcommand*\listfigurename{List of Figures}
\else
  \newcommand\listfigurename{List of Figures}
\fi
\ifdefined\listtablename
  \renewcommand*\listtablename{List of Tables}
\else
  \newcommand\listtablename{List of Tables}
\fi
\ifdefined\figurename
  \renewcommand*\figurename{Figure}
\else
  \newcommand\figurename{Figure}
\fi
\ifdefined\tablename
  \renewcommand*\tablename{Table}
\else
  \newcommand\tablename{Table}
\fi
}
\@ifpackageloaded{float}{}{\usepackage{float}}
\floatstyle{ruled}
\@ifundefined{c@chapter}{\newfloat{codelisting}{h}{lop}}{\newfloat{codelisting}{h}{lop}[chapter]}
\floatname{codelisting}{Listing}
\newcommand*\listoflistings{\listof{codelisting}{List of Listings}}
\makeatother
\makeatletter
\makeatother
\makeatletter
\@ifpackageloaded{caption}{}{\usepackage{caption}}
\@ifpackageloaded{subcaption}{}{\usepackage{subcaption}}
\makeatother
\usepackage{bookmark}
\IfFileExists{xurl.sty}{\usepackage{xurl}}{} % add URL line breaks if available
\urlstyle{same}
\hypersetup{
  pdftitle={Impact of working remotely on social wellbeing and productivity},
  pdfauthor={Amaury Chartier; Valentine Salvat; Thomas Blondel; Elisa Rigazzio},
  colorlinks=true,
  linkcolor={blue},
  filecolor={Maroon},
  citecolor={Blue},
  urlcolor={Blue},
  pdfcreator={LaTeX via pandoc}}


\title{Impact of working remotely on social wellbeing and productivity}
\author{Amaury Chartier \and Valentine Salvat \and Thomas
Blondel \and Elisa Rigazzio}
\date{2026-10-12}
\begin{document}
\maketitle
\begin{abstract}
This study explores how remote work affects productivity, social
well-being, and job satisfaction. Using the 2020 NSW Remote Working
Survey, we applied descriptive statistics and correlation analysis to
quantify attitudes toward remote work. Findings show mixed productivity
outcomes: most employees report gains up to 50\%, fewer others declines.
Collaboration indicators remain neutral, while reduced commuting
significantly increases personal and family time, supporting better
work--life balance. Most respondents prefer more remote work, suggesting
higher satisfaction overall. Limitations include self-reported data and
cross-sectional design. Results indicate organizations should maintain
flexible work arrangements and support social engagement strategies.
\end{abstract}


\section{Introduction}\label{introduction}

\subsection{Project Goals}\label{project-goals}

Our objective in this project is to understand how remote work,
following the COVID-19 pandemic, has changed firms' work practices.
During this period, many companies had no choice but to adopt remote
working practices, reshaping traditional work environments. The main
goal is to analyze how working remotely influences people's mental
health, job satisfaction, and performance levels. Using accurate data
from the Remote Working Survey (2020), we want to explore factors such
as communication satisfaction, work-life balance, social connection, and
productivity outcomes, in order to understand whether remote work
affects employees positively or negatively. By converting survey
responses into numerical values, we can quantify attitudes toward remote
work and compare groups such as gender, age, or employment type.

As we began working with the dataset, we refined our focus on the social
and psychological dimensions of remote work. To make the analysis easier
to interpret, we also transformed qualitative survey responses into
numerical values. And, in addition, we removed categories with very
small sample sizes to keep the dataset consistent and avoid unreliable
comparisons.

\subsection{Research Questions}\label{research-questions}

\begin{itemize}
\tightlist
\item
  How does remote work influence employees' overall productivity?
\item
  How does remote working affect employees' ability to maintain social
  connections and avoid feelings of isolation?
\item
  Has job satisfaction decreased or increased as a result of remote
  working conditions?
\item
  Which factors contribute the most to employees' overall satisfaction
  with remote work?
\end{itemize}

\subsection{Related Work}\label{related-work}

In this section, we discuss the main academic studies, methods, and
resources that guided our analysis of remote work and well-being.

\textbf{Domain literature}

A lot of studies have looked at how remote work affects people's
productivity, well-being, and social life, which helps us understand the
context of our own project.

For example, a systematic review by Oakman et al.~(2022) showed that
working from home can improve productivity and work--life balance, but
it can also lead to more isolation, communication problems, and mental
stress. These points match the variables we analyze in our dataset, such
as job satisfaction, social connection, and communication quality.
Another study by Correia et al.~(2024) investigated how research on
remote workers has evolved over the years. They found that psychological
factors like loneliness, emotional pressure, and reduced social
interaction are becoming increasingly important in academic discussions.
This supports our decision to focus mainly on the social and
psychological side of remote work instead of technical aspects.

Finally, a well-known experiment by Bloom et al.~(2013) showed that
remote work can significantly improve productivity when employees have
clear structures and good communication with their team. This connects
directly to our analysis, since we also look at collaboration,
communication, and satisfaction using the Remote Working Survey (2020).
Overall, these studies confirm that the social, psychological, and
performance-related impacts of remote work are important topics to
explore, and they strongly support the research questions we chose for
our project.

\textbf{Methodological references}

For our project, we used a few common data-science methods that are
usually applied in survey analysis. We started with simple descriptive
statistics to get a first idea of the patterns in the data, things like
averages, proportions, and basic comparisons between groups. We also
looked at correlations to see how different variables are related, for
example whether good communication or social well-being is linked to
higher productivity. Since many of our questions were answered with text
options (like ``strongly agree''), we converted these answers into
numbers using standard Likert-scale coding so that we could analyze them
properly.

On the technical side, we followed the Python methods shown in Azizi
(2025), which helped us clean the data and structure our exploratory
analysis in Google Colab. We mainly used pandas for organizing and
transforming the dataset (following McKinney, 2022), and
matplotlib/seaborn to create the visualizations with the help of
VanderPlas, 2016. These tools and references guided our process and
helped us follow good data-science practices while analyzing the Remote
Working Survey (2020).

\textbf{Course material}

The structure of our project is inspired by the material used in class,
especially the DSAS notes from Azizi (2025). These resources helped us
understand how to organize our data analysis, clean our dataset, and
apply basic Python techniques in Google Colab. With that said, we used
the same approach shown in the course for tasks like loading data,
creating new variables, handling missing values, and running exploratory
data analysis. The examples provided in the course made it easier for us
to use tools like pandas, matplotlib, and seaborn in a consistent way
throughout the project.

\textbf{Technical resources}

On the technical side, we mainly relied on a few well-known Python
resources to help us structure and run our analysis. McKinney's book
(2022) guided us with all the pandas-related tasks, such as cleaning the
dataset, creating new variables, and handling missing values. We also
used matplotlib and seaborn to create readable plots for our
visualizations by following VanderPlas's (2016) explanations.

\section{Data}\label{data}

\subsection{Sources}\label{sources}

The dataset used in this project is the 2020 Remote Working Survey, part
of the NSW Remote Working Survey series, publicly available through the
New South Wales Government's Open Data Portal. ``Data source:
\href{https://data.nsw.gov.au/data/dataset/nsw-remote-working-survey}{Remote
working survey 2020}'' The 2020 survey was conducted during August and
September 2020 and aimed to understand workers experiences and attitudes
toward remote and hybrid working following the first phase of the
COVID-19 pandemic to study the impact of remote work on professional and
personal well-being.

To be eligible, respondents had to:

\begin{itemize}
\item
  be employed NSW residents
\item
  have experience of remote working in their current job. After
  excluding unemployed individuals and those whose occupations cannot be
  performed remotely (dentists, cashiers, cleaners), the sample
  represents approximately 59\% of NSW workers.
\end{itemize}

This dataset corresponds to the 2020 wave of the NSW Remote Working
Survey. Although a second wave of the survey was collected in March and
April 2021, it is not included in the present analysis. The 2021 dataset
differs substantially in terms of survey structure, variable
definitions, and measurement scales, which would require extensive
re-harmonisation and could introduce inconsistencies across waves. To
preserve internal consistency and ensure methodological robustness, the
analysis therefore focuses exclusively on the 2020 dataset.

To provide additional contextual insight, an external dataset sourced
from Kaggle is introduced at the exploratory data analysis stage. This
dataset contains country-level COVID-19 indicators, including total
cases and deaths, and is used for descriptive cross-country comparison.
It is not merged with the survey data and serves only to contextualize
the remote working analysis.

\subsection{Description}\label{description}

The cleaned 2020 NSW Remote Working Survey dataset contains four main
types of variables: categorical, ordinal, numeric, and binary.

\begin{itemize}
\item
  The categorical variables describe qualitative characteristics of
  respondents and their employment context, such as Industry, Job\_type,
  Organisation\_Size, Household, and Years\_in\_job. These variables are
  stored as text and provide context for grouping and comparison across
  sectors or demographic profiles.
\item
  The ordinal variables represent ordered responses on Likert scales,
  reflecting opinions and perceptions about remote working. Variables
  such as Org\_encouraged\_remote\_last\_year,
  Collaboration\_remote\_last\_year, Org\_encouraged\_remote\_3\_months,
  and Collaboration\_remote\_3\_months are encoded numerically from 1
  (``Strongly disagree'') to 5 (``Strongly agree''), allowing for
  quantitative analysis of attitudes.
\item
  The numeric variables capture measurable quantities including time
  allocation, remote work proportions, and productivity. Examples
  include Age, Remote\_pct\_last\_year, Preferred\_remote\_last\_year,
  Productivity\_remote\_vs\_workplace, and several variables
  representing hours spent on commuting, working, and personal or
  domestic activities. These are stored as integers or floats, making
  them suitable for descriptive statistics and correlation analysis.
\item
  The binary variables (Gender, Managing\_position) indicate Male/Femal
  or Yes/No conditions, encoded as 0 and 1. These variables enable
  comparisons between distinct groups.
\end{itemize}

The dataset also contains several important pieces of metadata that
ensure its quality and usability for analysis. Each observation is
identified by a unique respondent code (Response\_ID), which guarantees
traceability and prevents duplication during data processing.

In addition, all variable names were standardized to concise,
descriptive identifiers (Org\_encouraged\_remote\_last\_year) to make
the variables easier to read, reuse, and analyze in statistical
software. This naming convention makes the analysis process simpler and
clearer throughout the project.

\begin{tcolorbox}[enhanced jigsaw, opacityback=0, leftrule=.75mm, colbacktitle=quarto-callout-note-color!10!white, colframe=quarto-callout-note-color-frame, rightrule=.15mm, coltitle=black, toprule=.15mm, toptitle=1mm, bottomtitle=1mm, breakable, titlerule=0mm, left=2mm, title=\textcolor{quarto-callout-note-color}{\faInfo}\hspace{0.5em}{Dataset Overview Template}, opacitybacktitle=0.6, arc=.35mm, bottomrule=.15mm, colback=white]

\begin{itemize}
\item
  \textbf{File used:} 2020\_rws-updated.csv
\item
  \textbf{Format:} CSV (comma-separated values)
\item
  \textbf{Encoding:} latin-1 (ISO-8859-1) (Remark: when loading the
  dataset in Python (Google Colab), attempting to read with utf-8 caused
  a UnicodeDecodeError due to typographic apostrophes and special
  characters. The correct parameter encoding latin1 was required to
  successfully load the data.)
\item
  \textbf{Memory usage:} approximately 7.46 MB
\item
  \textbf{Number of observations:} 1507 respondents (1370 rows after
  cleaning)
\item
  \textbf{Number of variables:} 73 columns (25 after cleaning)
\item
  \textbf{Time period:} August and September 2020
\item
  \textbf{Geographic coverage:} New South Wales, Australia
\item
  \textbf{Key variables:} Gender, Age, Job\_type, Organisation\_Size,
  Managing\_position, Remote\_pct\_last\_year,
  Preferred\_remote\_last\_year, Org\_encouraged\_remote\_last\_year,
  Collaboration\_remote\_last\_year, Productivity\_remote\_vs\_workplace
\end{itemize}

\end{tcolorbox}

\subsection{Loading Data}\label{loading-data}

Following best practices, the file is loaded using a relative path via
project\_root, ensuring that the document remains fully reproducible
regardless of the execution location. Because the original file
contained specific typographic characters that caused decoding issues
during import, the dataset is read using the latin-1 encoding.

After loading the file, several checks were performed to confirm correct
import: verification of the dataset dimensions, inspection of column
names and data types (df.info()), preview of the first rows to ensure
values were properly formatted. These steps guarantee that the dataset
is correctly imported and ready for subsequent exploratory analysis.

\begin{verbatim}
Dataset shape: 1507 rows × 73 columns

Data types:
Response ID                                                                                                                                                                                                                                                                                                int64
What year were you born?                                                                                                                                                                                                                                                                                   int64
What is your gender?                                                                                                                                                                                                                                                                                      object
Which of the following best describes your industry?                                                                                                                                                                                                                                                      object
Which of the following best describes your industry? (Detailed)                                                                                                                                                                                                                                           object
                                                                                                                                                                                                                                                                                                           ...  
Compare remote working to working at your employer’s workplace. Select the worst aspect of remote working for you - The number of hours  I work ; My work-life balance ; My on-the-job learning opportunities ; Managing my personal commitments ; My opportunities to socialise ; My mental wellbeing    object
Compare remote working to working at your employer’s workplace. Select the best aspect of remote working for you - The number of hours  I work ; My work-life balance ; My on-the-job learning opportunities ; My daily expenses ; My personal relationships ; My job satisfaction                        object
Compare remote working to working at your employer’s workplace. Select the worst aspect of remote working for you - The number of hours  I work ; My work-life balance ; My on-the-job learning opportunities ; My daily expenses ; My personal relationships ; My job satisfaction                       object
Compare remote working to working at your employer’s workplace. Select the best aspect of remote working for you - Managing my personal commitments ; My opportunities to socialise ; My mental wellbeing ; My daily expenses ; My personal relationships ; My job satisfaction                           object
Compare remote working to working at your employer’s workplace. Select the worst aspect of remote working for you - Managing my personal commitments ; My opportunities to socialise ; My mental wellbeing ; My daily expenses ; My personal relationships ; My job satisfaction                          object
Length: 73, dtype: object

First 5 rows:
\end{verbatim}

\phantomsection\label{load-data}
\begin{longtable}[]{@{}llllllllllllllllllllll@{}}
\toprule\noalign{}
& Response ID & What year were you born? & What is your gender? & Which
of the following best describes your industry? & Which of the following
best describes your industry? (Detailed) & Which of the following best
describes your current occupation? & Which of the following best
describes your current occupation? (Detailed) & How many people are
currently employed by your organisation? & Do you manage people as part
of your current occupation? & Which of the following best describes your
household? & ... & Compare remote working to working at your employer's
workplace. Select the best aspect of remote working for you - Managing
my family responsibilities ; My working relationships ; Preparing for
work and commuting ; Managing my personal commitments ; My opportunities
to socialise ; My mental wellbeing & Compare remote working to working
at your employer's workplace. Select the worst aspect of remote working
for you - Managing my family responsibilities ; My working relationships
; Preparing for work and commuting ; Managing my personal commitments ;
My opportunities to socialise ; My mental wellbeing & Compare remote
working to working at your employer's workplace. Select the best aspect
of remote working for you - Managing my family responsibilities ; My
working relationships ; Preparing for work and commuting ; My daily
expenses ; My personal relationships ; My job satisfaction & Compare
remote working to working at your employer's workplace. Select the worst
aspect of remote working for you - Managing my family responsibilities ;
My working relationships ; Preparing for work and commuting ; My daily
expenses ; My personal relationships ; My job satisfaction & Compare
remote working to working at your employer's workplace. Select the best
aspect of remote working for you - The number of hours I work ; My
work-life balance ; My on-the-job learning opportunities ; Managing my
personal commitments ; My opportunities to socialise ; My mental
wellbeing & Compare remote working to working at your employer's
workplace. Select the worst aspect of remote working for you - The
number of hours I work ; My work-life balance ; My on-the-job learning
opportunities ; Managing my personal commitments ; My opportunities to
socialise ; My mental wellbeing & Compare remote working to working at
your employer's workplace. Select the best aspect of remote working for
you - The number of hours I work ; My work-life balance ; My on-the-job
learning opportunities ; My daily expenses ; My personal relationships ;
My job satisfaction & Compare remote working to working at your
employer's workplace. Select the worst aspect of remote working for you
- The number of hours I work ; My work-life balance ; My on-the-job
learning opportunities ; My daily expenses ; My personal relationships ;
My job satisfaction & Compare remote working to working at your
employer's workplace. Select the best aspect of remote working for you -
Managing my personal commitments ; My opportunities to socialise ; My
mental wellbeing ; My daily expenses ; My personal relationships ; My
job satisfaction & Compare remote working to working at your employer's
workplace. Select the worst aspect of remote working for you - Managing
my personal commitments ; My opportunities to socialise ; My mental
wellbeing ; My daily expenses ; My personal relationships ; My job
satisfaction \\
\midrule\noalign{}
\endhead
\bottomrule\noalign{}
\endlastfoot
0 & 1 & 1972 & Female & Manufacturing & Food Product Manufacturing &
Clerical and administrative & Other Clerical and Administrative &
Between 20 and 199 & No & Couple with no dependent children & ... &
Managing my personal commitments & My opportunities to socialise &
Preparing for work and commuting & My working relationships & Managing
my personal commitments & The number of hours I work & My job
satisfaction & The number of hours I work & Managing my personal
commitments & My opportunities to socialise \\
1 & 2 & 1972 & Male & Wholesale Trade & Other Goods Wholesaling &
Managers & Chief Executives, General Managers and Legisla... & Between 1
and 4 & Yes & Couple with dependent children & ... & Preparing for work
and commuting & My working relationships & Preparing for work and
commuting & My working relationships & My work-life balance & My
on-the-job learning opportunities & My work-life balance & My on-the-job
learning opportunities & My personal relationships & My opportunities to
socialise \\
2 & 3 & 1982 & Male & Electricity, Gas, Water and Waste Services & Gas
Supply & Managers & Chief Executives, General Managers and Legisla... &
More than 200 & Yes & One parent family with dependent children & ... &
Managing my personal commitments & Preparing for work and commuting &
Preparing for work and commuting & Managing my family responsibilities &
The number of hours I work & My mental wellbeing & The number of hours I
work & My daily expenses & My mental wellbeing & My daily expenses \\
3 & 4 & 1987 & Female & Professional, Scientific and Technical Services
& Professional, Scientific and Technical Services & Professionals & ICT
Professionals & Between 20 and 199 & No & Couple with dependent children
& ... & Preparing for work and commuting & My opportunities to socialise
& My personal relationships & My working relationships & My work-life
balance & My on-the-job learning opportunities & My work-life balance &
My on-the-job learning opportunities & My personal relationships & My
job satisfaction \\
4 & 5 & 1991 & Male & Transport, Postal and Warehousing & Other
Transport & Managers & Specialist Managers & Between 5 and 19 & Yes &
Couple with no dependent children & ... & Managing my personal
commitments & My working relationships & Preparing for work and
commuting & My daily expenses & My work-life balance & My on-the-job
learning opportunities & My work-life balance & My on-the-job learning
opportunities & My opportunities to socialise & My job satisfaction \\
\end{longtable}

\subsection{Wrangling}\label{wrangling}

\subsubsection{General Transformations}\label{general-transformations}

Several preprocessing and wrangling steps were performed to prepare the
dataset for analysis. Before detailing each transformation.

All preprocessing steps described below are fully reproducible and
validated. Each transformation relies on deterministic Python operations
(such as replace(), rename(), astype(), or simple arithmetic), meaning
that re-running the same code on the raw dataset will always produce
identical results. After every transformation, checks such as head(),
value\_counts(), or describe() were used to validate that the
modifications were correctly applied and that the resulting values were
coherent (age ranges, Likert scales, or percentage conversions).

\textbf{Variable Duplicated and Variable Reduction}

Before any transformation, it is essential to verify that each
observation in the dataset is unique. Duplicate rows can bias the
analysis by over representing certain respondents or records.
Identifying and removing them ensures data integrity.

This operation can be rerun on the raw dataset at any stage of the
workflow, guaranteeing consistent detection of duplicated records. The
command returned an empty DataFrame, confirming that no duplicate rows
were present. Therefore, no observations were removed in this step.

In parallel, a large number of variables from the initial dataset were
removed during preprocessing in order to improve readability,
interpretability, and analytical relevance. This included variables such
as industry, current occupation, share of time spent remote working,
perceived best aspects of remote work (working hours, work-life balance,
job satisfaction), and perceived barriers to remote working
(organisational systems, workspace conditions, or management
discouragement). This was done by dropping selected columns as shown
below:

This step allowed the dataset to be reduced to a smaller and more
meaningful set of variables, while preserving all observations.

\textbf{Rename Columns}

Renaming the original survey questions into shorter and clearer variable
names was necessary to improve readability and make the dataset easier
to work with.

\textbf{Standard Name Organisation Size}

The Organisation\_Size variable was recoded by replacing the long
original text categories with shorter, standardised. To make grouping
and comparison more intuitive.

\textbf{Binary Variables}

The variables Gender and Managing\_position, the original text responses
were converted into binary categories to make them suitable for
statistical analysis and group comparisons. Respondents who selected
``Rather not say'' for gender were removed, as this category represented
only two individuals and could not form a meaningful subgroup. Gender
was then encoded as 0 = Male and 1 = Female, while Managing\_position
was encoded as 0 = No and 1 = Yes, indicating whether the respondent
supervises others. This transformation produces clean, consistent, and
analysis-ready binary variables that can be easily used in descriptive
statistics, visualisations, and modelling.

\textbf{Variable Ages}

The variable Age, the dataset originally reported the respondent's year
of birth. This information was converted into a more interpretable age
variable by subtracting the birth year from 2020, the year the survey
was conducted. For example, a respondent born in 1985 becomes 2020 −
1985 = 35 years old. Expressing this information directly as age is more
intuitive and easier to interpret in descriptive statistics,
comparisons, and visualizations.

\textbf{Variable Years in Job}

The variable Years\_in\_job, the original responses were long text
categories describing tenure intervals. These were simplified into
shorter and more readable labels: ``5+'', ``5-'', and ``1-''. This
transformation keeps the original meaning while making the variable
easier to interpret, compare, and visualise in tables and plots.

\textbf{Variable Likert Scale Mapping}

The Likert-scale variables, this transformation allows these subjective
perceptions to be analysed quantitatively. The four variables related to
organisational support and collaboration were mapped using this scale.
Any missing responses were replaced with the neutral value 3,
corresponding to ``Neither agree nor disagree'', to keep these
observations in the dataset while avoiding bias from missing attitudes.

\textbf{Variables Working Remotely}

The variables describing the percentage of time spent or preferred
working remotely, the original responses (``Less than 10\% of my
time''\ldots{}``100\% - All of my time''). These were first converted
into numeric percentage values using a mapping dictionary. All four
percentage-related variables were processed in the same way. Before
applying the mapping, non-breaking spaces () and extra whitespace were
removed from the strings to avoid parsing issues. After the text values
were mapped to numeric percentages (``80\%'' → 80), the values were
converted into proportions between 0 and 1 by dividing by 100.

For example: 80 becomes 0.80 50 becomes 0.50 0 becomes 0.00

This final step creates clean numerical variables that can be easily
averaged, compared, or visualised in the analysis.

\textbf{Variables Productivity Cleaning}

The variable Productivity\_remote\_vs\_workplace, the survey responses
(I'm 20\% more productive when I work remotely'' or ``I'm 10\% less
productive''). These text responses were converted into clean numeric
values using a custom parsing function. The mapping works as follows:
Statements indicating more productive remotely return a positive
percentage (``20\% more productive'' → +20). Statements indicating less
productive remotely return a negative percentage (``10\% less
productive'' → --10). Statements indicating no difference return 0. This
transformation results in a numeric scale where positive values mean
higher productivity when working remotely, negative values reflect lower
productivity, and zero indicates no change. This allows the variable to
be analysed quantitatively, averaged across groups, or used in
visualisations.

\subsubsection{Spotting Mistakes and Missing
Data}\label{spotting-mistakes-and-missing-data}

Before conducting the analysis, the dataset was reviewed to identify
missing values, inconsistencies, and unusually small categories. This
ensures that only reliable and interpretable data are used in the
following steps.

\textbf{Identified missing data}

\begin{itemize}
\item
  Most variables contained very few missing values, mainly in
  attitudinal questions where some respondents simply did not answer.
\item
  Additional missingness appeared when converting textual inputs
  (percentages or productivity statements) into numeric formats, entries
  that could not be parsed were intentionally converted to NaN.
\item
  The inspection of category sizes showed that some groups were
  extremely small, such as the ``Rather not say'' gender category (2
  respondents), which was removed because it cannot support meaningful
  analysis.
\end{itemize}

\textbf{Approach to handling missing data}

\begin{itemize}
\item
  Deletion was applied when missingness or category size was extremely
  small and analytically useless.
\item
  Imputation with a neutral value (3 = Neither agree nor disagree) was
  used for missing Likert-scale answers to preserve observations without
  creating bias.
\item
  Conversion to numeric with errors=``coerce'' was used for percentage
  and productivity variables, producing valid NaN values when entries
  could not be interpreted.
\end{itemize}

Because missingness was limited and mostly isolated to subjective
questions, more complex methods were not necessary.

\textbf{Future handling of small or irrelevant categories}

\begin{itemize}
\tightlist
\item
  If additional categories, during the analysis progress, are found to
  be too small to contribute meaningfully, they will be removed,
  flagged, or when conceptually appropriate grouped together with
  similar categories to preserve statistical power.
\end{itemize}

\textbf{Future variable selection}

\begin{itemize}
\tightlist
\item
  Some variables will ultimately explain the effects of remote work on
  health, productivity, or work life balance better than others.
  Variables that show no meaningful correlation or explanatory power in
  later stages of the project will also be removed to keep the analysis
  focused, interpretable, and relevant.
\end{itemize}

\subsubsection{Listing Anomalies and
Outliers}\label{listing-anomalies-and-outliers}

A detailed inspection of the numeric variables, using both summary
statistics and histogram visualisations, revealed several anomalies and
potential outliers in the dataset.

\textbf{Detected anomalies}

\begin{itemize}
\item
  Age variable: Two respondents appear with an age of 120 years, which
  is biologically impossible and indicates a clear data entry error.
  This type of anomaly commonly occurs when the birth year is
  mistyped---for example, entering 1900 instead of 2000, or 2005 instead
  of 1920---which produces unrealistic age values. These observations
  will therefore be removed, while other extreme but plausible ages
  (such as 75 or 83) are retained at this stage. The minimum value (19)
  is plausible for entry-level workers.
\item
  Domestic\_hours\_workplace: A single observation of --1 hour is
  impossible and indicates a recording error.
\item
  Working and commuting time variables: Extremely high records were
  observed, such as 23 hours of work in a day or 10--12 hours of
  commuting.
\end{itemize}

While unlikely, these may represent exceptional cases (long-distance
travel, extended shifts). They are retained unless later analysis shows
they distort results.

\begin{itemize}
\item
  Commute\_hours\_remote: Values up to 12 hours on remote days are
  implausible and likely due to misinterpretation or input mistakes.
\item
  Productivity variable: Extreme values (+50\%, --50\%) appear in the
  data but remain plausible since the question explicitly asked
  respondents to report percentage differences.
\end{itemize}

\textbf{Approach to handling outliers}

\begin{itemize}
\item
  Outliers were evaluated using:

  \begin{itemize}
  \item
    Visual inspection (histograms from univariate EDA)
  \item
    Summary statistics (min/max, interquartile ranges)
  \item
    Domain knowledge (negative hours, impossible ages)
  \end{itemize}
\item
  Following best practices

  \begin{itemize}
  \item
    Impossible values (age = 120, domestic hours = --1) will be removed
    before modelling.
  \item
    Extreme but plausible behaviours (very long workdays) are kept
    unless they later bias model results.
  \item
    Additional outliers identified during the analysis phase may be
    removed, flagged, or grouped depending on their relevance and
    impact.
  \end{itemize}
\end{itemize}

Outliers are not always errors; some reveal meaningful variability in
remote work habits. The chosen approach maintains data integrity while
ensuring that the analysis focuses on realistic, interpretable patterns.

The following EDA sections will further analyse these variables to
understand their patterns and relationships.

\section{EDA}\label{eda}

\subsection{Univariate Analysis}\label{univariate-analysis}

Examine each variable individually to understand its distribution,
central tendency, and spread.

From this code we got the full results of all our variables in terms of
distribution sorted by occurrence. The relevant information we gathered
from these are the followings:

\begin{verbatim}
Unable to display output for mime type(s): text/html
\end{verbatim}

\begin{verbatim}
Unable to display output for mime type(s): text/html
\end{verbatim}

\begin{itemize}
\tightlist
\item
  Age: we have some outliers for the age variable with 2 people 120
  years old that needs to be treated
\end{itemize}

\begin{verbatim}

--- Gender ---
Gender
0    766
1    604
Name: count, dtype: int64
\end{verbatim}

\begin{itemize}
\tightlist
\item
  Gender: good balance (604 / (604+766) = 44\% of women)
\end{itemize}

\begin{verbatim}

--- Years_in_job ---
Years_in_job
5+    725
5-    494
1-    151
Name: count, dtype: int64
\end{verbatim}

\begin{itemize}
\tightlist
\item
  Years in job: sample is quite experienced with only 11\% with less
  than 1 year of experience.
\end{itemize}

\begin{verbatim}

--- Productivity_remote_vs_workplace ---
Productivity_remote_vs_workplace
 0     400
 50    199
 20    191
 30    173
 10     91
-20     85
-10     85
 40     74
-30     40
-50     26
-40      6
Name: count, dtype: int64
\end{verbatim}

\begin{itemize}
\tightlist
\item
  Productivity remote vs workplace: we can see that for most of them
  productivity is either equivalent, or they even gained in productivity
  (46\% gained between 20 to 50\% of productivity).
\end{itemize}

\subsection{Distribution Plot}\label{distribution-plot}

Concretely, if we look at the distribution of the change in productivity
with this code :

\begin{verbatim}
Unable to display output for mime type(s): text/html
\end{verbatim}

\begin{verbatim}
Skewness change in productivity:
-0.1398603974975266
\end{verbatim}

The histogram we obtain from the code shows an important part of 0
meaning for a lot of people working remotely doesn't affect their
productivity. Looking at the skewness (-0.13), we can deduce that data
is significantly more on the right of the graph, meaning that people are
more productive being at home for work.

\subsection{Bivariate Analysis}\label{bivariate-analysis}

\begin{verbatim}
Unable to display output for mime type(s): text/html
\end{verbatim}

In this Violin plot, we can understand that the change in productivity
is not really influenced by the size of the company. Nevertheless, we
can observe for small enterprises that a negative change in productivity
is even rarer. We can guess that people in small businesses don't depend
too much on working with peers and with teams, so going to the workplace
is less necessary than in big companies and communications facilitated.

\begin{verbatim}
Unable to display output for mime type(s): text/html
\end{verbatim}

We wanted to look at the distribution of the managing positions based on
the gender and we could see that almost 60\% of men are managers
compared to only 40\% for women so this is factored to take into
consideration when giving conclusions.

\subsection{Correlation/Multivariate
Analysis}\label{correlationmultivariate-analysis}

\begin{verbatim}
Unable to display output for mime type(s): text/html
\end{verbatim}

\begin{verbatim}
Unable to display output for mime type(s): text/html
\end{verbatim}

can see on those 2 pie charts representing the distribution of the day
based on the time in hours working, commuting, for personnal and family,
and domestic. We can see that the time of work is sensibly the same for
remote and workplace (50\% compared to 48.8\%). What is important here
is to see that time to commute is divided by 3 when remote working so it
gives more time for domesting and personnal/family time going from
39.6\% of your day to 46\%. This difference shows a better work/life
balance for employees when remotely working. This could affect mental
health positively.

\subsection{Key Findings}\label{key-findings}

\textbf{Univariate Analysis} - The sample is professionally experienced:
only 11\% have experience of less than 1 year. - There is a correct
balance between genders: 44\% of the sample is composed by women. -
Productivity is either not affected or positively impacted by remote
work: 46\% improved their productivity through remote work.

\textbf{Bivariate Analysis} Working on place or remotely does not affect
small businesses productivity, from which we can conclude
small-structure workers does not depend too much on working with peers
and teams, and going physically to workplace is less necessary than in
big companies and communications facilitated.

Manager jobs are composed of men at a 60\% rate whereas only 40\% are
women, a factor we must take into consideration when giving conclusions.

\textbf{Correlation/Multivariate Analysis} According to the distribution
of the day based on the time in hours working, commuting, for personal
and family, and domestic, we can observe that the time of work is
sensibly the same remotely and physically at work (50\% compared to
48.8\%). Furthermore, time to commute is divided by 3 when working
remotely, which gives more time for domestic activities and
personal/family time going from 39.6\% of daytime to 46\%. This
difference shows a better work/life balance for employees through remote
work, an element that could positively affect mental health.\\
There is a positive correlation of 0.14 between productivity and working
remotely. However, this correlation coefficient remains positively low.

\subsection{Findings and Discussion}\label{findings-and-discussion}

\textbf{Perceived productivity (Productivity\_remote\_vs\_workplace)}

The distribution remains wide but not excessively dispersed. Both
positive and negative self-reported productivity changes appear, with no
clear dominant direction.

\textbf{Remote work share (Remote\_pct\_*)}

Clear multimodal distribution: - a large cluster at 0 (never remote), -
a midpoint cluster around 0.5 (hybrid), - and a peak at 1.0 (fully
remote).

The regression plot shows a weak positive trend between remote work
percentage and perceived productivity.

Remote\_pct\_*: meaning all the columns that start with Remote\_pct\_ --
in our context, remote\_pct\_last\_year and remote\_pct\_last\_3\_months
variables.

\textbf{Hours commute / work / personal / domestic}

These variables exhibit high variability, and under the classical IQR
test, some show large numbers of outliers (Working\_hours\_workplace:
271, Commute\_hours\_remote: 142).

Commute\_hours\_remote is essentially zero for most respondents, whereas
workplace commute shows very high values for a subgroup.

**Satisfaction / Preferences (Preferred\_remote\_*)**

A significant proportion of respondents indicate a preference for higher
levels of remote work --- suggesting overall satisfaction with remote
arrangements.

\textbf{Demographics}

Age is concentrated within typical working-age ranges but includes two
implausible values around 120, considered invalid extremes under both
IQR methods. Gender distribution shows moderate imbalance.

\textbf{Outliers \& cleaning (3×IQR method)}

Outlier counts highlight substantial structural variability in the
dataset. Using the classical 1.5×IQR rule would remove more than 40\% of
the sample, which risks discarding legitimate behavioral diversity.
Using the 3×IQR rule instead focuses only on filtering clear data errors
or implausible extreme values (ages \textgreater100, commute hours
\textgreater10, negative preferred remote percentages). With this more
cautious threshold and a tolerance of zero outliers per row, the dataset
is reduced from 1370 to 1196 rows, meaning only obviously invalid
extreme records were removed. This approach preserves representativity
while still improving data quality.

\subsection{Interpretation (per
finding)}\label{interpretation-per-finding}

Perceived productivity Histograms and boxplots show the distribution of
Productivity\_remote\_vs\_workplace is still wide after cleaning.

The regression line suggests a slight positive correlation between
remote work percentage and productivity. The mean appears small relative
to the variance, indicating high heterogeneity in productivity
experiences. Some individuals report substantial productivity gains,
others substantial losses.

Through these observations, we directly answer this question: ``How does
remote work influence employees' overall productivity?'', and the answer
is: effects differ strongly across individuals; no universal impact
exists. It is consistent with Bloom et al.~(2015) and COVID-era studies
showing mixed or context-dependent productivity outcomes.

The limitations are composed of biases and complexity-lacking
interpretation such as Self-reported measure bias and univariate-only
interpretation.

Thus, organisations should adopt flexible and individualized policies,
and further multivariate analysis should identify moderators (job type,
household structure, management role\ldots).

\textbf{Job satisfaction / preferences}

Preferred\_remote\_* shows many respondents preferring more remote work.
Preference likely reflects higher satisfaction with remote work
conditions relative to on-site conditions.

It directly informs remote work increases job satisfaction for many
workers and addresses: ``Has job satisfaction increased or decreased due
to remote working?'': univariate evidence points to increased
satisfaction for many workers. Preference does not equal measured job
satisfaction; there is a potential selection bias. Hybrid working models
and flexible remote-work options should be explored.

\textbf{Social connection / collaboration} Likert variables for
collaboration (Collaboration\_remote\_*) are centered around neutral
values (Likert value: 3). At an aggregate level, respondents do not
report a clear decline or improvement in collaboration while working
remotely.

It addresses the question: ``How does remote working affect employees'
ability to maintain social connections and avoid isolation?''. Digital
collaboration tools may offset loss of informal interactions and
individual variability likely remains. Likert responses are proxies; no
direct psychological isolation is measured.

However, a preference-based bias might be taken into consideration: as
remote work seems more desirable for most respondents, they might answer
in a way that avoids remote work discreditation. Organizations should
focus interventions on subgroups reporting lower collaboration.

Structural factors (commute and personal time) Remote commute is almost
zero; remote work yields increase in personal, family, and domestic
hours. Savings in commute time and reallocation toward personal
activities are plausible drivers of increased satisfaction with remote
work. We can extrapolate this job and free-time balance to an enhanced
mental health and less stressed psychology.

To ``Which factors contribute most to remote-work satisfaction?'', we
can thus conclude that Time savings and increased autonomy are strong
candidates, which aligns with research showing commute reduction
increases well-being and perceived control over time. However, the
trends are univariate only; further multivariate modeling might be
needed. Policies should preserve gains in personal time and support
work-life balance.

\subsection{Overall answers to research
questions}\label{overall-answers-to-research-questions}

\textbf{Productivity}

Remote work has a mixed and highly heterogeneous effect on productivity;
the average effect is not clearly positive or negative.

\textbf{Social connection / isolation}

Aggregate responses are neutral and no evidence of widespread isolation
or deterioration of collaboration is detected.

\textbf{Job satisfaction}

Preference patterns imply that for most workers, job satisfaction has
increased under remote arrangements.

\textbf{Drivers of satisfaction}

The strongest univariate candidates are: - reduced commute, - increased
personal/family time, - greater schedule flexibility.

\section{Analysis}\label{analysis}

\subsection{Methods}\label{methods}

\textbf{Remote Work Intensity and Productivity}

This section examines the relationship between remote work intensity and
self-reported productivity change using both graphical analysis and
linear regression models.

\textbf{Exploratory relationship}

Figure X plots productivity change against the share of time spent
working remotely, together with a fitted linear regression line. The
slope of the line is positive, and the Pearson correlation coefficient
is approximately 0.14, indicating a weak positive association between
remote work intensity and productivity change. Individuals who work
remotely more frequently tend, on average, to report slightly higher
productivity. However, the magnitude of this association is limited and
should not be interpreted as causal.

\textbf{Baseline regression model}

To formalise this relationship, productivity change is first modelled
using a simple ordinary least squares (OLS) regression with remote work
intensity as the only explanatory variable: \[
\text{Productivity}_i =
\beta_0
+ \beta_1 \,\text{RemoteShare}_i
+ \varepsilon_i
\]

\begin{verbatim}
Unable to display output for mime type(s): text/html
\end{verbatim}

This baseline model captures the raw relationship between remote work
intensity and productivity without accounting for any additional
individual or organisational characteristics. The estimation results
show that the coefficient on remote work intensity is positive and
statistically significant. Moving from no remote work to full remote
work is associated with an increase of approximately 11--12 percentage
points in reported productivity. Despite this statistically significant
effect, the explanatory power of the model remains very low, with an
R-squared of around 0.02. This indicates that remote work intensity
alone explains only a very small share of the variation in productivity
changes across individuals.

\textbf{Productivity by remote work intensity groups}

To complement the continuous analysis, productivity changes are also
examined by grouping individuals into remote work intensity categories
(0--20\%, 20--50\%, 50--80\%, and 80--100\%). This binned representation
allows for a clearer comparison of productivity distributions across
levels of remote work.

The results show that median productivity gains increase once remote
work exceeds 20\% of working time and stabilise around +20\% for medium
to high levels of remote work intensity. Mean productivity changes
follow a similar pattern, rising from approximately +10\% in the lowest
group to nearly +20\% in the highest group.

However, the distributions overlap substantially across groups, with
wide dispersion in all categories and both positive and negative
productivity changes observed at every level of remote work intensity.
This highlights considerable individual heterogeneity and reinforces the
conclusion that remote work intensity alone is insufficient to explain
productivity outcomes.

\begin{verbatim}
Unable to display output for mime type(s): text/html
\end{verbatim}

\textbf{Extended regression model with controls}

The analysis is then extended by including additional explanatory
variables that capture individual and organisational characteristics:

\[
\text{Productivity}_i =
\beta_0
+ \beta_1 \,\text{RemoteShare}_i
+ \beta_2 \,\text{Age}_i
+ \beta_3 \,\text{OrganisationSize}_i
+ \beta_4 \,\text{ManagingPosition}_i 
\n + \beta_5 \,\text{OrgEncouragement}_i
+ \beta_6 \,\text{Collaboration}_i
+ \varepsilon_i
\]

These variables were selected based on exploratory analysis and
correlation checks, as they showed meaningful associations with
productivity change and are directly related to working conditions under
remote work. After including these controls, the coefficient on remote
work intensity remains positive and statistically significant, although
slightly smaller, at around 9.6. Importantly, the explanatory power of
the model improves: the R-squared increases from approximately 0.02 in
the baseline model to about 0.05 in the extended specification. This
increase indicates that organisational context and collaboration quality
account for an additional share of the variation in productivity
changes. Among the control variables, perceived collaboration quality
when working remotely shows a strong positive association with
productivity, while organisational encouragement of remote work is
negatively associated. Age and managing position do not exhibit
statistically significant effects once other factors are controlled for.

\section{Interpretation}\label{interpretation}

Overall, the results convey a consistent message across graphical
analysis and regression models. Remote work intensity is positively
associated with productivity change, but the relationship is weak in
magnitude and explains only a limited fraction of productivity
differences. The increase in R-squared when adding organisational and
collaboration-related variables confirms that productivity outcomes
under remote work are influenced by multiple factors. While remote work
intensity plays a role, individual heterogeneity and organisational
conditions appear to be more important drivers of productivity changes.

\section{Conclusion}\label{conclusion}

\subsection{Summary}\label{summary}

In this project, we analyzed how remote work affects employees'
productivity, social well-being, and job satisfaction using the Remote
Working Survey (2020), which includes 1,507 respondents. We investigated
our four research questions using descriptive statistics,
visualizations, and correlation analysis after cleaning the dataset and
translating qualitative responses into numerical values. Our results
show that the effect of remote work on productivity is highly
heterogeneous. While many respondents reported no change, the
productivity distribution displayed a slight negative skew (--0.13),
meaning productivity gains were more common than losses. Productivity
did not vary significantly by organisation size, and the correlation
between remote-work percentage and productivity change was positive but
weak (r = 0.14), suggesting that working remotely more often is
associated with a small increase in perceived productivity.

The clearest pattern we found was the improvement in work--life balance.
Remote work nearly eliminated commuting time, and our comparison showed
that personal and family time increased from 39.6\% to 46\% of the day
on average when employees worked from home. This extra time probably
improved mental health and could be responsible for the overall high
desire for remote work. Indicators of collaboration and connection on
the social side were centered around neutral responses, showing that
employees' feeling of social connection was neither greatly improved nor
harmed by distant work. Overall, our research indicates that working
remotely can boost well-being and productivity, mainly through time
savings and better daily time management, but its effects on social
interaction seem to be less clear. These findings help us better
understand how employees' lives are shaped by remote work and directly
address the objectives of our project.

\subsection{Limitations}\label{limitations}

Although our project provides meaningful insights, it also has several
limitations that should be acknowledged. First, the dataset comes from a
single region (New South Wales, Australia) and only reflects the year
2020, when remote work was still relatively new due to the COVID-19
pandemic. This context may not represent long-term or post-pandemic
habits, which limits how generalizable our findings are to other
geographic areas, industries, or years. The data also contained several
extreme outliers (for example, two respondents aged 120), as well as
categories with very small sample sizes such as ``rather not say'' for
gender. We removed these categories to avoid inconsistent comparisons,
but doing so slightly reduces the diversity of our sample.

\textbf{Comparative Analysis of the Impact of COVID-19 (United States,
India, France, Brazil, Australia)}

The comparative analysis highlights substantial disparities in the
evolution and impact of the COVID-19 pandemic across the five selected
countries. The United States records the highest number of total cases,
followed by India, France, and Brazil, whereas Australia clearly stands
out with a significantly lower number of reported infections. With
respect to mortality, a similar pattern is observed. The United States
and Brazil exhibit the highest death tolls among the sample, while
France occupies an intermediate position. In contrast, Australia
displays a markedly lower mortality level, indicating a comparatively
limited health impact of the pandemic. The analysis of active cases,
defined as the number of individuals currently infected (total cases
minus recoveries and deaths), reveals that active infections remained
relatively high in the United States and Brazil at the time of data
collection.

Interpretation of the Australian Case

Based on our interpretation of the graphical results, Australia appears
to have been relatively less affected by the COVID-19 pandemic,
particularly in terms of mortality and active infections. Several
structural and geographical factors may help explain this outcome.
First, Australia has a comparatively younger population, and younger age
groups were statistically less exposed to severe forms of the disease.
Furthermore, the country is characterized by a large territory combined
with a low population density, which naturally limits close
interpersonal contact and reduces the speed of viral transmission. In
addition, Australia's geographical isolation likely delayed the initial
introduction of the virus and facilitated the implementation of strict
border control measures, which played a critical role during the early
stages of the pandemic. Finally, early public health interventions and
the overall efficiency of the healthcare system may also have
contributed to limiting severe cases and fatalities.

\begin{figure}

\begin{minipage}{0.50\linewidth}

\centering{

\includegraphics[width=1\linewidth,height=\textheight,keepaspectratio]{images/Australia_vs_countries.png}

}

\subcaption{\label{fig-after}comparison}

\end{minipage}%

\caption{\label{fig-comparison}The emotional journey of a data scientist
debugging their code}

\end{figure}%

From a methodological standpoint, our study mostly uses correlation
analysis and descriptive statistics, which help in finding patterns but
cannot allow us to prove causation. For example, even if we discovered a
slight positive correlation (r = 0.14) between the percentage of remote
work and productivity change, this does not show that remote work raises
productivity. These results could be impacted by additional unobserved
factors including home environment, digital technologies, management
assistance, or individual preferences. In addition, a number of
important factors such as productivity, teamwork, and satisfaction are
self-reported, which may create bias because workers may exaggerate or
understate their experiences.

\subsection{Future Work}\label{future-work}

\begin{itemize}
\tightlist
\item
  \textbf{Extend dataset:} ``Include the 2021 wave of the Remote Working
  Survey and perform longitudinal comparisons to track changes over
  time.''
\item
  \textbf{Enhance analytical approach:} ``Use advanced statistical
  models like multiple regression to identify key predictors of
  productivity and satisfaction.''
\item
  \textbf{Broaden contextual factors:} ``Add variables on digital tools,
  communication habits, and mental health indicators for richer
  insights.''
\item
  \textbf{Integrate mixed data types:} ``Combine quantitative survey
  data with qualitative responses to capture nuanced experiences of
  remote work in other regions not only Australia.''
\end{itemize}

\subsection{References}\label{references}

\begin{itemize}
\tightlist
\item
  NSW Government. (2020). Remote Working Survey 2020 Dataset. Retrieved
  from https://data.nsw.gov.au/data/dataset/nsw-remote-working-survey
  (Used to describe data source, survey methodology, and metadata.)
\item
  Pandas Documentation. (n.d.). Pandas User Guide. Retrieved from
  https://pandas.pydata.org/docs/ (Referenced for operations such as
  rename(), replace(), map(), astype(), errors=``coerce'',
  value\_counts(), info().)
\item
  Seaborn Documentation. (n.d.). Seaborn: Statistical Data
  Visualization. Retrieved from https://seaborn.pydata.org/ Matplotlib
  Documentation. (n.d.). Matplotlib: Visualization with Python.
  Retrieved from https://matplotlib.org/stable/ (Used for histograms,
  boxplots, and univariate analysis.)
\item
  Quarto Documentation. (n.d.). Quarto Guide. Retrieved from
  https://quarto.org/ (Referenced for callouts, section structure, code
  blocks, and PDF/HTML formatting best practices.)
\item
  Joshi, A., Kale, S., Chandel, S., \& Pal, D. K. (2015). Likert Scale:
  What it is \& How to Use It. Retrieved from
  https://www.researchgate.net/publication/281874183\_Likert\_Scale\_What\_it\_is\_and\_How\_to\_Use\_It
  (Implicitly used to justify converting textual modalities to numeric
  values 1--5.)
\item
  Charalampous, M., Grant, C. A., Tramontano, C., \& Michailidis, E.
  (2022). Investigating the Role of Remote Working on Employees'
  Performance and Well-Being: An Evidence-Based Systematic Review.
  International Journal of Environmental Research and Public Health,
  19(19), 12373. Retrieved from
  https://pmc.ncbi.nlm.nih.gov/articles/PMC9566387/
\item
  García-Sánchez, E., \& García-Sánchez, I. M. (2024). Remote workers'
  well-being: Are innovative organizations really concerned? A
  bibliometrics analysis. Journal of Innovation \& Knowledge, 9(4),
  100313. Retrieved from
  https://www.sciencedirect.com/science/article/pii/S2444569X24001343
\item
  Bloom, N., Liang, J., Roberts, J., \& Ying, Z. J. (2013). Does Working
  from Home Work? Evidence from a Chinese Experiment. National Bureau of
  Economic Research Working Paper No.~18871. Retrieved from
  https://www.nber.org/papers/w18871
\end{itemize}




\end{document}
